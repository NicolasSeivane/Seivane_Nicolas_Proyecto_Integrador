\documentclass[a4paper,10pt]{book}

\usepackage[T1]{fontenc}
\usepackage[utf8]{inputenc}
%\usepackage{mathtools}
\usepackage{graphicx}
\usepackage[margin=1in]{geometry}
%\usepackage[spanish]{babel}
\usepackage[hidelinks]{hyperref}
\usepackage{fancyhdr}
\usepackage{incgraph,tikz}
\usepackage{caption}
\usepackage{subcaption}
\usepackage{amsfonts}
\usepackage{babel}
\usepackage{color}
\usepackage{listings}
\usepackage{float}
\usepackage{tikz}
\usepackage{adjustbox}
\usepackage{natbib}
\usepackage{subcaption}
\usepackage{cleveref}
\usepackage{xcolor}
\usepackage{titlepic}

\usepackage{listings}
\lstset{ %
basicstyle=\footnotesize,       % the size of the fonts that are used for the code
numbers=none,                   % where to put the line-numbers
numberstyle=\footnotesize,      % the size of the fonts that are used for the line-numbers
stepnumber=1,                   % the step between two line-numbers. If it is 1 each line will be numbered
numbersep=10pt,                  % how far the line-numbers are from the code
backgroundcolor=\color{white},  % choose the background color. You must add \usepackage{color}
showspaces=false,               % show spaces adding particular underscores
showstringspaces=false,         % underline spaces within strings
showtabs=false,                 % show tabs within strings adding particular underscores
frame=single,           % adds a frame around the code
tabsize=4,          % sets default tabsize to 2 spaces
captionpos=b,           % sets the caption-position to bottom
breaklines=true,        % sets automatic line breaking
breakatwhitespace=false,    % sets if automatic breaks should only happen at whitespace
escapeinside={\%*}{*)}          % if you want to add a comment within your code
}

\usetikzlibrary{arrows}

\captionsetup[subfigure]{subrefformat=simple,labelformat=simple}

\renewcommand{\contentsname}{\'Indice General}
\renewcommand{\listfigurename}{\'Indice de Figuras}
\renewcommand{\listtablename}{\'Indice de Tablas}
\renewcommand{\lstlistingname}{Salida}
\renewcommand\tablename{Tabla}
\renewcommand{\figurename}{Figura}
\renewcommand\thesubfigure{(\alph{subfigure})}
\renewcommand{\baselinestretch}{1.2} 

\makeatletter
\def\verbatim{\small\@verbatim \frenchspacing\@vobeyspaces \@xverbatim}
\makeatother

\pagestyle{fancy}

\restylefloat{table}

\title{Título}
\author{Autor: \\ Tutor:}
\date{Fecha \\ Universidad Nacional de Hurlingham}
\titlepic{\vspace{12cm}\includegraphics[width=0.15\textwidth]{logo.jpg}}

\begin{document}

\maketitle


\tableofcontents
\listoffigures
\listoftables

\chapter*{Resumen}

\chapter{Introducción}


Describir el problema que se desea resolver

\section{Motivación}

Explicar porqué estudiamos este problema, para qué sirve, cuál es el impacto y en qué áreas.


\section{Estado del Arte}


En esta sección se realiza una descripción de algunos de los métodos más importantes existentes en la bibliografía describiendo el problema y el método utilizado por cada autor. Se cita la bibliografía. 
 
\section{Conjuntos de datos Utilizados}
Se describen los datos utilizados y de donde fueron extraídos, si corresponde. En caso de que los datos sean propios, describir como fueron tomados

\chapter{Descripción de los Métodos Utilizados}

Describir los métodos utilizados, si corresponde. 



\chapter{Resultados}

Mostrr los resultados obtenidos utilizando gráficos, tablas, figuras, etc

\chapter{Conclusiones}

Explicar que aprendimos con la realización de este trabajo. Qué nos muestran los resultados. 

\bibliographystyle{plain}
\bibliography{References}

\end{document}
